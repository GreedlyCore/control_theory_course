\chapter{Общие выводы}
\label{ch:chap6}

В этой работе были рассмотрены некоторые объекты управления, внутри которых скрывались типовые звенья с определёнными параметрами, описывающими их физическую суть. 
Чтобы изучить каждый из объектов мы нашли его передаточную функцию, и с помощью неё смогли посмотреть на временные характеристики системы - её поведение при импульсном и ступенчатом воздействии. 
После мы взглянули на частотную передаточную функцию, и смогли узнать частотные характеристики системы - АЧХ, ФЧХ в линейном и Логарифмическом масштабе. Эти результаты были проделаны с помощью моделирования $\textrm{Matlab}$ и аналитических расчётов.

Использовал связку \textit{Live-script + Matlab}, там же можно взглянуть на графики и код, в \href{https://github.com/GreedlyCore/control_theory_course}{репозитории} можно найти исходники. 
\endinput