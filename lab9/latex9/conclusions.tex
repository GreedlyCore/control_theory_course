\chapter{Общие выводы}
\label{ch:chap6}

В ходе выполнения лабораторной работы был рассмотрен синтез регулятора с заданной степенью устойчивости путём задания этой степени коэффициентом $\alpha$ и решения линейный матричных уравнений типа Ляпунова или 
методом решения матричного уравнения Риккати. Для более плавного и оптимального управления был также опробован регулятор с качественной экспоненциальной устойчивостью, который показал ожидаемые результаты.
Также был синтезирован наблюдатель с заданной степени сходимости, алгоритм вычисления сводился к тем же численным методам, были рассмотрены различные коэффцициенты $\alpha$ его сходимости.

Синтезированные компоненты системы проверялись при помощи компьютерного моделирования, наблюдатель успешно сходился к истинной системе, 
а регулятор успешно приводил в положение равновесия во всех случаях с разной степенью энергозатрат и оптимальности в виде сочетания перерегулирования между временем переходного процесса.

Использовал связку \textit{Live-script + Matlab}, все исходные материалы, использованные в работе можно найти  в \href{https://github.com/GreedlyCore/control_theory_course}{репозитории}.

Здесь также активно начинаем использовать внешнюю библиотеку для матлаба - \text{CVX}, которая помогает при решении \text{LMI} и многих других проблем.
\endinput