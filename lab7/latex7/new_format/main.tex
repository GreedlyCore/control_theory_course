\documentclass[a4paper,14pt,oneside,openany]{article}


\usepackage{listings}
\usepackage{indentfirst} 

\usepackage{fontspec}     
\setmainfont{CMU Serif}   
\newfontfamily\cyrillicfonttt{DejaVu Sans Mono} 
\usepackage{polyglossia}
\setdefaultlanguage{russian}
\setotherlanguage{english}

\renewcommand{\lstlistingname}{Листинг}
% Отображение страницы
\usepackage{geometry} % размеры листа и отступов
\geometry{
	left=12mm,
	top=25mm,
	right=15mm,
	bottom=17mm,
	marginparsep=0mm,
	marginparwidth=0mm,
	headheight=10mm,
	headsep=7mm,
	nofoot}
\usepackage{afterpage,fancyhdr} % настройка колонтитулов
\pagestyle{fancy}
\fancypagestyle{style}{ % создание нового стиля style
	\fancyhf{} % очистка колонтитулов
	\fancyhead[LO, RE]{} % название документа наверху
	\fancyhead[RO, LE]{\leftmark} % название section наверху
	\fancyfoot[RO, LE]{\thepage} % номер страницы справа внизу на нечетных и слева внизу на четных
	\renewcommand{\headrulewidth}{0.25pt} % толщина линии сверху
	\renewcommand{\footrulewidth}{0pt} % толцина линии снизу
}
\fancypagestyle{plain}{ % создание нового стиля plain -- полностью пустого
	\fancyhf{}
	\renewcommand{\headrulewidth}{0pt}
}
\fancypagestyle{title}{ % создание нового стиля title -- для титульной страницы
	\fancyhf{}
	\fancyhead[C]{{\footnotesize \Large
			Министерство образования и науки Российской Федерации\\
			Федеральное государственное автономное образовательное учреждение высшего образования
	}}
	\fancyfoot[C]{{\large 
			Санкт-Петербург, \the\year
	}}
	\renewcommand{\headrulewidth}{0pt}
}

% Математика
\usepackage{amsmath, amsfonts, amssymb, amsthm} % Набор пакетов для математических текстов
%\usepackage{dmvnbase} % мехматовский пакет latex-сокращений
\usepackage{cancel} % зачеркивание для сокращений
% Рисунки и фигуры
\usepackage{graphicx}        
% \usepackage[unicode]{hyperref} % Was: \usepackage[unicode,pdftex]{hyperref}
\usepackage{wrapfig, subcaption} % вставка фигур, обтекая текст
\usepackage{caption} % для настройки подписей
\captionsetup{figurewithin=none,labelsep=period, font={small,it}} % настройка подписей к рисункам
% Рисование
\usepackage{tikz} % рисование
\usepackage{circuitikz}
\usepackage{pgfplots} % графики
% Таблицы
\usepackage{multirow} % объединение строк
\usepackage{multicol} % объединение столбцов
% Остальное
\usepackage{enumitem} % нормальное оформление списков
\usepackage{awesomebox}
\usepackage{tabularx}
\usepackage{hyperref} % гиперссылки
\usepackage{cleveref} % гиперссылки2

% Настройки стиля для листинга
\lstdefinestyle{mystyle}{
backgroundcolor=\color{gray!10!white},
basicstyle=\ttfamily,
language=Python,
numbers=left,
numberstyle=\small,
numbersep=5pt,
breaklines=true,
showstringspaces=false,
keywordstyle=\color{blue},
commentstyle=\color{green!40!black},
stringstyle=\color{green!40!black},
frame=single,
rulecolor=\color{black},
tabsize=2,
basicstyle=\small\ttfamily,
}


\setlist{itemsep=0.15cm,topsep=0.15cm,parsep=1pt} % настройки списков
% Теоремы, леммы, определения...
\theoremstyle{definition}
\newtheorem{Def}{Определение}
\newtheorem*{Axiom}{Аксиома}
\theoremstyle{plain}
\newtheorem{Th}{Теорема}
\newtheorem{Lem}{Лемма}
\newtheorem{Cor}{Следствие}
\newtheorem{Ex}{Пример}
\theoremstyle{remark}
\newtheorem*{Note}{Замечание}
\newtheorem*{Solution}{Решение}
\newtheorem*{Proof}{Доказательство}

\DeclareMathOperator{\sinc}{sinc}
\DeclareMathOperator{\Si}{Si}

% Свои команды
\newcommand{\comb}[1]{\left[\hspace{-4pt}\begin{array}{l}#1\end{array}\right.\hspace{-5pt} } % совокупность уравнений
% Титульный лист
\usepackage{csvsimple-l3}


\newcommand\tab[1][1cm]{\hspace*{#1}}

% https://tex.stackexchange.com/questions/2705/typesetting-column-vector
% Column Vector Macros
% EXAMPLE: \colvec{5}{a}{b}{c}{d}{e}
\newcount\colveccount
\newcommand*\colvec[1]{
        \global\colveccount#1
        \begin{pmatrix}
        \colvecnext
}
\def\colvecnext#1{
        #1
        \global\advance\colveccount-1
        \ifnum\colveccount>0
                \\
                \expandafter\colvecnext
        \else
                \end{pmatrix}
        \fi
}

% https://tex.stackexchange.com/questions/39051/typesetting-a-row-vector
% Row Vector Macros
% EXAMPLE: \rowvec{a,b,c,d,e}
\newcommand*{\rowvec}[1]{\left( #1\right)}

% \ExplSyntaxOn
% \clist_new:N \l_feq_vector_clist
% \NewDocumentCommand{\feqvector}{O{\\}mO{b}}{
%   \clist_set:Nn \l_feq_vector_clist {#2} % Set the list
%   \begin{#3matrix}
%   \clist_use:Nn \l_feq_vector_clist {#1} % show it with separator from #1 (\\)
%   \end{#3matrix}
% }
% \ExplSyntaxOff

%%% Вставляем по очереди все содержательные части документа %%%
\thispagestyle{empty}

\begin{center}
    МИНИСТЕРСТВО НАУКИ И ВЫСШЕГО ОБРАЗОВАНИЯ \\ РОССИЙСКОЙ ФЕДЕРАЦИИ

    \vspace{20pt}

    Университет ИТМО

    \vspace{20pt}

    Факультет систем управления и робототехники
\end{center}

\vfill

\begin{center}
    ОТЧЁТ \\  
    по лабораторной работе  №6, вариант - 2 \\
    \textit{Линейные системы автоматического управления}

    \vspace{20pt}

    по теме: \\
    \uppercase{Критерий Найквиста и системы с запаздыванием}
\end{center}

\vfill

\noindent Студент: \\
\textit{Группа R3336 \hfill Поляков А.А.}


    \vspace{20pt}

    \noindent Предподаватель: \\
    \textit{к.т.н., доцент \hfill  Пашенко А.В.}

\vfill

\begin{center}
    Санкт-Петербург \\ 2024
\end{center}

\begin{document}
\titlePage                                          % Титульник
\pagestyle{style}
\newpage

\tableofcontents                                   % Автособираемое оглавление

\chapter{Модальные регулятор}
\label{ch:chap1}
\section{Условие задачи}

Необходимо рассмотреть систему:
$$
  \dot{x} = Ax + Bu
$$ и выполнить следующие шаги:

\begin{itemize}
\item   Найти собственные числа матрицы $A$ и определить управляемость каждого из них. 
Сделать вывод об управляемости и стабилизируемости системы.
\item Построить схему моделирования системы, замкнутой регулятором $u = Kx$.
\item Рассмотреть предложенные желаемые спектры замкнутой системы $(A+BK)$ и определить, 
какие из них достижимы, а какие нет. Обосновать выбор.
\item Для каждого из достижимых спектров вашего варианта:
  \begin{itemize}
    \item Найти соответствующую матрицу регулятора $K$, приводящий спектр 
    замкнутой системы к желаемому.
    \item Определить собственные числа матрицы замкнутой системы $(A+BK)$ и 
    сравнить с желаемым спектром в подтверждение корректности синтеза регулятора.
    \item  Выполнить компьютерное моделирование и построить графики формируемого 
    регулятором управления $u(t)$ и вектора состояния замкнутой системы $x(t)$
     при начальных условиях $x(0) = \begin{bmatrix} 1 & 1 & 1 \end{bmatrix}^T$.
  \end{itemize}
\item Сопоставить полученные результаты компьютерного моделирования для рассмотренных спектров, 
оценить возможные сравнительные преимущества и недостатки  каждого из них.
\end{itemize}

\section{Решение задачи}

Параметры для объекта:
$$
  A = \begin{bmatrix}
  12 & -1 & 14 \\
  6 & 0 & 6 \\
  -6 & -2 & -8 
  \end{bmatrix} \tab
  B = \begin{bmatrix}
    11 \\ 7 \\ -7 
  \end{bmatrix}
$$

\subsection{Исследование управляемости системы}

Найдём собственные числа матрицы $A$:
$$
    \lambda_{1,2} = 3 \pm 3i, \tab \lambda_3 = -2
$$

Вычислим матрицу Хаутуса для каждого собственного числа:
$$
    H_1 = \begin{bmatrix}
          A - \lambda_1 I & B   
          \end{bmatrix} = 
    \begin{bmatrix}
    9-3i & -1 & 14 & 11 \\  
    6 & -3-3i & 6 & 7 \\  
    -6 & -2 & -11 -3i & -7   
    \end{bmatrix}
$$
$$
rank(H_1) = 3
$$
Значит собственное число $\lambda_1$ является управляемым, если ранг его матрицы Хаутуса равняется порядку системы.
$$
    H_2 = \begin{bmatrix}
          A - \lambda_2 I & B   
          \end{bmatrix} = 
    \begin{bmatrix}
      9+3i & -1 & 14 & 11 \\  
      6 & -3+3i & 6 & 7 \\  
      -6 & -2 & -11+3i & -7   
    \end{bmatrix}
$$
$$
rank(H_2) = 3
$$
Аналогично, собственное число $\lambda_2$ является управляемым.
$$
    H_3 = \begin{bmatrix}
          A - \lambda_3 I & B   
          \end{bmatrix} = 
    \begin{bmatrix}
      14 & -1 & 14 & 11 \\
      6 & 2 & 4 & 7 \\
      -6 & -2 & -4 & -7
    \end{bmatrix}
$$
$$
$$
$$
rank(H_3) = 2
$$
Последнее собственное число $\lambda_3$ уже не управляемое. Значит система по критерию Хаутуса будет не полностью управляемой. 

Однако система будет стабилизируемой, потому что это все неустойчивые (их нет) собственные числа у нас будут управляемые.

\subsection{Достижимые спектры}
Нам дан следующий набор спектров замкнутной системы $(A+BK)$:
$$
  \begin{aligned}
    \sigma_1 = \{-1, -1, -1\} \\
    \sigma_2 = \{-2, -2, -2\} \\
    \sigma_3 = \{-1, -10, -100\} \\
    \sigma_4 = \{-2, -20, -200\} \\
    \sigma_5 = \{-1, -1-3i, -1+3i\} \\
    \sigma_6 = \{-2, -2-6i, -2+6i\} 
  \end{aligned}
$$

Из него не все спектры мы сможем использовать как достижимы у матрицы замкнутой $(A+BK)$ на регулятор системы, потому что наша система не полностью управляема.
Ограничивает нас $\lambda_3=-2$ (не управляемое собственное число), которое нельзя будет изменить никак с помощью регулятора, поэтому оно обязано остаться одним из собственных чисел $(A+BK)$ после замыкания.
После такого отбора останутся следующие спектры, и они уже будут достижимыми:

$$
  \begin{aligned}
    \sigma_2 = \{-2, -2, -2\} \\
    \sigma_4 = \{-2, -20, -200\} \\
    \sigma_6 = \{-2, -2-6i, -2+6i\} 
  \end{aligned}
$$

\subsection{Первый спектр}
$$
    \sigma_1 = \{-2, -2, -2\} \\
$$

Найдём матрицу регулятора K, приводящий спектр замкнутой системы к желаемому. Для её получения воспользуемся следующей системой:
$$
\begin{cases}
  AP - PG = BY, \\
  K = -YP^{-1}
\end{cases}
$$ и убедимся в том, что следующие условия также более-менее выполняются:
$$
  \begin{cases}
    \sigma(A) \cap \sigma(G) = \emptyset, \\
    (A,B) - \text{управляема, то всегда будет существовать } P \\
    (Y,G) - \text{ наблюдаема } 
  \end{cases}
$$
Матрицы $G,Y$ - выборочные и вспомогательные при синтезе регулятора:
\begin{itemize}
  \item Выбираем $G\in\mathbb{R}^{n\times n}$ с желаемым спектром $\sigma(G)$
  \item Выбираем $Y\in\mathbb{R}^{m\times n}$ такую, чтобы пара $(Y, G)$ была наблюдаема
  \item Ищем $P\in\mathbb{R}^{n\times n}$ как решение уравнения Сильвестра $AP - PG = BY$
  \item Вычисляем коэффициенты регулятора: $K = -YP^{-1}$
\end{itemize}

В нашем случае пара $(A,B)$ - лишь стабилизируема, но это не значит, что регулятор нельзя синтезировать, пока что это лишь значит, что
решение может быть не единственным или/и вырожденным. Остальные два условия мы выполним умным выбором $Y,G$:
$$
G = \begin{bmatrix}
    -2  &  1  & 0 \\
     0  & -2  & 1 \\
     0  &  0  & -2 
\end{bmatrix} \tab Y = \begin{bmatrix}
  1 & 1 & 1
\end{bmatrix}
$$

Проделав вычисления, получим следующие коэффициенты:
$$
  K = \begin{bmatrix}
    1.09 & -2.08 & 1.06
  \end{bmatrix}
$$
Тогда получим следующие собственные числа $(A+BK)$:
$$
    \lambda_{1,2,3} = -2
$$
Что совпадает с исходно заданным спектром, а значит контроллер мы синтезировали верно.

Для проведения моделирования была составлена следующая схема:
\begin{figure}[ht]
  \centering
  \includegraphics[width=1.0\textwidth]{model_controller.png}
  \caption{Модель с модальным регулятором}
\end{figure}

С помощью неё построим графики управления $u(t)$ от регулятора и вектора состояния замкнутой системы $x(t)$
при начальных условиях $x(0) = \begin{bmatrix} 1 & 1 & 1 \end{bmatrix}^T$:
\newpage
\begin{figure}[ht]
  \centering
  \includegraphics[width=0.8\textwidth]{ctrl_u1.png}
  \caption{Сигнал управления}
\end{figure}
\begin{figure}[ht]
  \centering
  \includegraphics[width=0.8\textwidth]{ctrl_x1.png}
  \caption{Состояние системы}
\end{figure}

\newpage
\subsection{Второй спектр}
$$
    \sigma_2 = \{-2, -20, -200\} \\
$$

Найдём матрицу регулятора K, воспользуемся следующей системой:
$$
\begin{cases}
  AP - PG = BY, \\
  K = -YP^{-1}
\end{cases}
$$ Для этого с умом выберем $Y,G$:
$$
G = \begin{bmatrix}
    -2  &  0  & 0 \\
     0  & -20  & 0 \\
     0  &  0  & -200 
\end{bmatrix} \tab Y = \begin{bmatrix}
  1 & 1 & 2
\end{bmatrix}
$$

Проделав вычисления, получим следующие коэффициенты:
$$
  K = \begin{bmatrix}
    425.89 & -387.02 & 314.51
  \end{bmatrix}
$$
Тогда получим следующие собственные числа $(A+BK)$:
$$
    \lambda_1 = -2, \tab \lambda_2 = -20, \tab \lambda_3 = -200
$$
Что совпадает с исходно заданным спектром, а значит контроллер мы синтезировали верно.

Построим графики управления $u(t)$  и вектора состояния $x(t)$:
\begin{figure}[ht]
  \centering
  \includegraphics[width=0.8\textwidth]{ctrl_u2.png}
  \caption{Сигнал управления}
\end{figure}
\begin{figure}[ht]
  \centering
  \includegraphics[width=0.8\textwidth]{ctrl_x2.png}
  \caption{Состояние системы}
\end{figure}

\newpage
\subsection{Третий спектр}
$$
  \sigma_3 = \{-2, -2-6i, -2+6i\} 
$$

Найдём матрицу регулятора K, воспользуемся следующей системой:
$$
\begin{cases}
  AP - PG = BY, \\
  K = -YP^{-1}
\end{cases}
$$ Для этого с умом выберем $Y,G$:
$$
G = \begin{bmatrix}
    -2  &  0  & 0 \\
     0  & -2  & 6 \\
     0  &  -6  & -2 
\end{bmatrix} \tab Y = \begin{bmatrix}
  1 & 1 & 0
\end{bmatrix}
$$

Проделав вычисления, получим следующие коэффициенты:
$$
  K = \begin{bmatrix}
    4.45 & -5.09 & 3.32
  \end{bmatrix}
$$
Тогда получим следующие собственные числа $(A+BK)$:
$$
    \lambda_1 = -2, \tab \lambda_{2,3} = -2 \pm 6i
$$
Что совпадает с исходно заданным спектром, а значит контроллер мы синтезировали верно.

Построим графики управления $u(t)$  и вектора состояния $x(t)$:
\newpage
\begin{figure}[ht]
  \centering
  \includegraphics[width=0.8\textwidth]{ctrl_u3.png}
  \caption{Сигнал управления}
\end{figure}
\begin{figure}[ht]
  \centering
  \includegraphics[width=0.8\textwidth]{ctrl_x3.png}
  \caption{Состояние системы}
\end{figure}

\subsection{Сравнение выбора собственных чисел для синтеза регулятора}

В первом случае мы выбрали относительно небольшие устойчивые моды, поэтому регулятор в целом корректировал систему плавно и с небольшим переругулированием.

Во втором случае мы решили ускорить процесс управления и многократно увеличили моды, поэтому управление действовало аггресивно, но привело
систему в нулевую позицию очень быстро. Однако в реальном мире таким управлением воспользоваться вряд ли удастся - нам помешают физические ограничения по току, 
потому иначе просто двигатель сможет сгореть от такого резкого и высокого напряжения, или просто такое управление переполнит буфер памяти контроллера. В общем, такое управление довольно рисковано.

В треьем случае в спектр вошла комплексно сопряжённая мода, поэтому система сходится с небольшими колебаниями, но вполне плавно, с небольшим перерегулированием, потому что вещественная часть этих мод не слишком велика.




\subsection{Вывод}

Исследование системы задания показало, что стабилизируемой системой мы можем управлять с разным "характером" сходимости, и не важна, что она не полностью управляема. 
Для управления мы успешно синтезировали модальный регулятор с помощью уравнения Сильвестра, желаемые спектры и спектр матрицы $(A+BK)$ - сошлись, что свидетельствует о корректном синтезе.

\endinput
\chapter{Коэффициент усиления}
\label{ch:chap2}

В соответствии с моим вариантом:
$$
    i=j=2, \tab W_1(s) = \frac{s-2}{s^2+6s+5}, \tab W_2(s) = \frac{-9s^3+16s^2-6s}{10s^3+12s^2+5s+1}
$$
Необходимо добавить к каждой функции коэффициент усиления $k > 0$.

\section{Передаточная функция $W_1$}
Для $k=1$:
$$
W_1(s) = \frac{1(s-2)}{s^2+6s+5},
$$
Рассчитаем полюса: $\lambda_{1,2} = \{-5, -1 \}$, значит разомкнутая система будет устойчива. 

Построим для неё годографы, с разными $k$:
\begin{figure}[ht]
    \centering
    \includegraphics[width=0.7\textwidth]{nyquist_task21_object1.png}
    \caption{Годограф Найквиста для разомкнутой системы, $k=1$}
\end{figure}
\begin{figure}[ht]
    \centering
    \includegraphics[width=0.7\textwidth]{nyquist_task22_object1.png}
    \caption{Годограф Найквиста для разомкнутой системы, $k=2$}
\end{figure}
\begin{figure}[ht]
    \centering
    \includegraphics[width=0.7\textwidth]{nyquist_task23_object1.png}
    \caption{Годограф Найквиста для разомкнутой системы, $k=3$}
\end{figure}
\begin{figure}[ht]
    \centering
    \includegraphics[width=0.7\textwidth]{nyquist_task24_object1.png}
    \caption{Годограф Найквиста для разомкнутой системы, $k=6$}
\end{figure}

Выходит, что коэффициент $k$ влияет на кривую годографа, расширяя её вдоль мнимой оси и влево по оси действительной части, в конце концов 
доходя критической точки $(-1,0)$.

Годограф делает обороты по часовой стрелки, а значит когда он дойдёт до критической точки, то замкнутая система приобретёт дополнительный несточивый полюс, когда коэффиент будет примерно $k > 2.5$ (критерий Найквиста).
Петля внутри не перемешается левее нуля при любом $k$, а значит мы максимум получим только один неустойчивый полюс, при $k > 2.5$.

Теперь проверим по критерию Гурвица это предположение:
$$
    W_{1,closed} = \frac{W_1}{1+W_1} = \frac{k(s-2)}{s^2 + (6+k)s + (5-2k)}
$$

$$
    \begin{cases}
        6+k > 0 \\
        5-2k > 0
    \end{cases} \to
    \begin{cases}
        k > -6 \\
        2.5 > k
    \end{cases}
$$
Но отрицательные значения мы не рассматриваем, поэтому по критерию Гурвица система будет асимптотически устойчива при $0 < k < 2.5$,
так как до этого значения годограф не захватывает точку $(-1, 0)$, а значит не добавляет дополнительный полюс.
\newpage
\subsection{Частотные характеристики}
Построим ФЧХ,АЧХ для нашей системы при $k=1$, не имеет смысла строить для других $k$, ибо они лишь будут масштабировать АЧХ, фазовые сдвиги при этом будут оставаться теми же.
\begin{figure}[ht]
    \centering
    \includegraphics[width=0.7\textwidth]{freq_ampl2_closed1.png}
    \caption{АЧХ для разомкнутой системы, $k=1$}
\end{figure}
\begin{figure}[ht]
    \centering
    \includegraphics[width=0.7\textwidth]{freq_phase2_closed1.png}
    \caption{ФЧХ для разомкнутой системы, $k=1$}
\end{figure}

График ФЧХ начинается со сдвига в $180^\circ$, и как этой частоте $\omega_{crit}$ у нас будет располагаться ближайшая точка от годографа к критической. А значит амплитуда для этой частоты:
$$
\frac{1}{A_3} = A(\omega_{crit}) = 0.4
$$
$$
A_3 = \frac{1}{0.4} = 2.5
$$
Значит, получается, что запас амплитуды равен критическому значению коэффициента $k_{crit}$.

\newpage
\subsection{Переходные функции}
\begin{figure}[ht]
    \centering
    \includegraphics[width=0.7\textwidth]{step_responce21_closed1.png}
    \caption{Переходная функция для замкнутой системы, $k=1$}
\end{figure}
\begin{figure}[ht]
    \centering
    \includegraphics[width=0.7\textwidth]{step_responce22_closed1.png}
    \caption{Переходная функция для замкнутой системы, $k=2$}
\end{figure}
\begin{figure}[ht]
    \centering
    \includegraphics[width=0.7\textwidth]{step_responce23_closed1.png}
    \caption{Переходная функция для замкнутой системы, $k=3$}
\end{figure}
\begin{figure}[ht]
    \centering
    \includegraphics[width=0.7\textwidth]{step_responce24_closed1.png}
    \caption{Переходная функция для замкнутой системы, $k=6$}
\end{figure}

Можно заметить, что при значениях $k < 2.5$ - замкнутая система действительно устойчива, а при $k > 2.5$ - неустойчивая уже.
Разомкнутая система при любом $k$ будет оставаться устойчивой, потому что не имеет ни одного правого корня.

\newpage
\section{Передаточная функция $W_2$}
Для $k=1$:
$$
W_2(s) = \frac{-9s^3+16s^2-6s}{10s^3+12s^2+5s+1}
$$
Рассчитаем полюса: $\lambda_{1,2,3} \approx \{-0.68, -0.26\pm 0.28j \}$, значит разомкнутая система будет устойчива. 

\newpage
Построим для неё годографы, с разными $k$:
\begin{figure}[ht]
    \centering
    \includegraphics[width=0.7\textwidth]{nyquist_task21_object2.png}
    \caption{Годограф Найквиста для разомкнутой системы, $k=1$}
\end{figure}
\begin{figure}[ht]
    \centering
    \includegraphics[width=0.7\textwidth]{nyquist_task22_object2.png}
    \caption{Годограф Найквиста для разомкнутой системы, $k=3$}
\end{figure}
\newpage
\begin{figure}[ht]
    \centering
    \includegraphics[width=0.7\textwidth]{nyquist_task23_object2.png}
    \caption{Годограф Найквиста для разомкнутой системы, $k=10$}
\end{figure}
\begin{figure}[ht]
    \centering
    \includegraphics[width=0.7\textwidth]{nyquist_task26_object2.png}
    \caption{Годограф Найквиста для разомкнутой системы, $k=0.2$}
\end{figure}
\newpage
\begin{figure}[ht]
    \centering
    \includegraphics[width=0.7\textwidth]{nyquist_task24_object2.png}
    \caption{Годограф Найквиста для разомкнутой системы, $k=0.6$}
\end{figure}
\begin{figure}[ht]
    \centering
    \includegraphics[width=0.7\textwidth]{nyquist_task27_object2.png}
    \caption{Годограф Найквиста для разомкнутой системы, $k=0.9$}
\end{figure}

Выходит, что коэффициент $k$ влияет на кривую годографа, расширяя её вдоль мнимой и вещественной оси. Уже при при $0.9 < k < 1$ мы начинаем получать 4 оборота по часовой стрелке, а значит +4 неустойчивых полюса для замкнутой системы, это  диапазон $k$  мы далее уточним при аналитике.
При бОльших $k > 1$ мы получаем +3 дополнительных нейстойчивых полюса для замкнутой системы, а при $k < 0.9$  - наша критическая точка и вовсе покидает годограф, он прилично сжимается. 

В итоге мы получили три диапазона для $k$, которые надо уточнить и проверить.
Чтобы узнать $k_{crit}$, посмотрим на критерий Гурвица:

$$
    W_{1,closed} = \frac{W_1}{1+W_1} = \frac{k(-9s^3 + 16s^2 -6s)}{(10-9k)s^3 + (12+16k)s^2 + (5-6k)s + 1 }
$$
Получим следующую систему:
$$
    \begin{cases}
        10-9k > 0 \\
        12+16k > 0 \\
        5-6k > 0 \\
        (12+16k)(5-6k) > (10-9k)
    \end{cases} \to
    -0.64 < k < 0.815
$$
Но отрицательные значения мы не рассматриваем, поэтому по критерию Гурвица система будет асимптотически устойчива при $0 < k < 0.815$,
так как до этого значения годограф не захватывает точку $(-1, 0)$, а значит не добавляет дополнительные неустойчивые полюса.
\newpage
\subsection{Частотные характеристики}
Построим ФЧХ,АЧХ для нашей системы при $k=1$, не имеет смысла строить для других $k$, ибо они лишь будут масштабировать АЧХ, фазовые сдвиги при этом будут оставаться теми же.
\begin{figure}[ht]
    \centering
    \includegraphics[width=0.7\textwidth]{freq_ampl2_closed2.png}
    \caption{АЧХ для разомкнутой системы, $k=1$}
\end{figure}
\begin{figure}[ht]
    \centering
    \includegraphics[width=0.7\textwidth]{freq_phase2_closed2.png}
    \caption{ФЧХ для разомкнутой системы, $k=1$}
\end{figure}

График ФЧХ начинается со сдвига в $630^\circ$ и снижается до $180^\circ$ асимптотическим образом. Значит в $180^\circ$ мы никогда не попадём, поэтому
ищем другой критический отрезок, у нас здесь остаётся только $540^\circ$. 

Однако стоит заранее остановить наш порыв найти запас по амплитуде, поскольку его нельзя определить для системы, которая уже при $k=1$ в замкнутом виде неустойчива. 
Поэтому с помощью частотных характеристик нам нельзя определить диапазон коэффицциента П-регулятора $k$. 

Тогда с помощью $\textrm{allmargin}$ получим список критических коэффициентов П-регулятора:
$$
    k_{max1} \approx 0.8156, \tab k_{max2} \approx 1.1111
$$

Мы уже нашли через критерий Гурвицца $k_{max1}$, а вместе со вторым коэффииентом мы теперь можем обозначить диапазоны для $k$:
\begin{enumerate}
    \item $k\in(0.00 ; 0.815)$ - замкнутая система будет иметь 0 неустойчивых полюсов.
    \item $k\in(0.815 ; 1.1111)$ - замкнутая система будет иметь 4 неустойчивый полюса (изначально система была устойчива). 
    \item $k\in(1.1111 ; +\infty)$ - замкнутая система будет иметь 3 неустойчивый полюса (изначально система была устойчива). 
\end{enumerate}

% В нём $\omega_{crit} \approx 0.031879$, тогда $A(\omega_{crit}) \approx 1.22901$
% $$
% A_3 = \frac{1}{A(\omega_{crit})} \approx 0.814
% $$
% Значит, получается, что запас амплитуды примерно равен критическому значению коэффициента $k_{crit}$. Я уверен, что они должны равняться, но просто нужно шаг уменьшить, нам и такой точности будет достаточно.



\newpage
\subsection{Переходные функции}
\begin{figure}[ht]
    \centering
    \includegraphics[width=0.7\textwidth]{step_responce21_closed2.png}
    \caption{Переходная функция для замкнутой системы, $k=1$}
\end{figure}
\begin{figure}[ht]
    \centering
    \includegraphics[width=0.7\textwidth]{step_responce22_closed2.png}
    \caption{Переходная функция для замкнутой системы, $k=3$}
\end{figure}

\newpage
\begin{figure}[ht]
    \centering
    \includegraphics[width=0.7\textwidth]{step_responce23_closed2.png}
    \caption{Переходная функция для замкнутой системы, $k=10$}
\end{figure}
\begin{figure}[ht]
    \centering
    \includegraphics[width=0.7\textwidth]{step_responce24_closed2.png}
    \caption{Переходная функция для замкнутой системы, $k=0.6$}
\end{figure}
\newpage
\begin{figure}[ht]
    \centering
    \includegraphics[width=0.7\textwidth]{step_responce26_closed2.png}
    \caption{Переходная функция для замкнутой системы, $k=0.2$}
\end{figure}
\begin{figure}[ht]
    \centering
    \includegraphics[width=0.7\textwidth]{step_responce27_closed2.png}
    \caption{Переходная функция для замкнутой системы, $k=0.9$}
\end{figure}

Можно заметить, что при значениях $0 < k < 0.815$ - замкнутая система действительно устойчива, а при $k > 0.815$ - неустойчивая уже, причём с разным количество неустойчивых полюсов.
Разомкнутая система при любом $k$ будет оставаться устойчивой, потому что не имеет ни одного правого корня.



\endinput
\chapter{Автономный генератор}
\label{ch:chap3}
\section{Постановка задачи}
В этом задании нам нужно добиться желаемого выхода, в случае второго варианта он будет таковым:
$$
g_{wanted}(t) = g_w(t) = cos(-2t) + e^{6t}sin(5t)
$$
А получить желаемый сигнал нам нужно посредством подбора параметров $A, C, x(0)$ для системы вида:
$$
\begin{cases}
    \dot{x} = Ax, \\
    g = Cx;
\end{cases}
$$
Выход системы мы рассматриваем при свободном движении, и он должен будет совпасть с $g_w(t)$.

Проверку наших матриц с желаемым сигналом мы будем осуществлять с помощью следующей структурной схемы:
\begin{figure}[ht]
    \centering
    \includegraphics[width=1\textwidth]{scheme_system3.png}
	\caption{Структурная схема - проверка}
\end{figure}


\newpage
\section{Восстановление матриц}
Свободному движению систему соответствует следующая система:
$$
\begin{cases}
    x_{free}(t) = e^{At}x(0), \\
    y_{free}(t) = Ce^{At}x(0);
\end{cases}
$$, где матрицы $C,x(0)$ - параметры системы:
$$
    C = \feqvector[&]{c_1, c_2, c_3, c_4}, \tab x(0) = \feqvector{a_1, a_2, a_3, a_4}
$$
Пойдём методом "Жордана", для этого посмотрим какие характерестические корни соответствуют решению $g_w(t)$:
$$
\lambda_{1,2} = \pm2i, \tab \lambda_{3,4}=6\pm5i
$$, где первая пара корней соответствует косинусу, а вторая - экспонента с синусом. Тогда составим следующую Жорданову матрицу по корням:
$$
A = \begin{bmatrix}
      0 & +2 & 0 & 0  \\
      -2 & 0 & 0 & 0 \\
      0 & 0 & 6 & 5 \\
      0 & 0 & -5 & 6
    \end{bmatrix}
$$
Тогда возьмём матричную экспоненту, а после домножим на начальные условия:
$$
e^{At}x(0) = exp\bigg(\begin{bmatrix}
    0 & +2 & 0 & 0  \\
    -2 & 0 & 0 & 0 \\
    0 & 0 & 6 & 5 \\
    0 & 0 & -5 & 6
  \end{bmatrix}t\bigg) \feqvector{a_1,a_2,a_3,a_4}
$$
Упростим выражение:
$$
\begin{aligned}
    e^{At}x(0) = \begin{bmatrix}
        cos(-2t) & sin(-2t) & 0 & 0  \\
        -sin(-2t) & cos(-2t) & 0 & 0 \\
        0 & 0 & e^{6t}cos(5t) & e^{6t}sin(5t) \\
        0 & 0 & -e^{6t}sin(5t) & e^{6t}cos(5t)
      \end{bmatrix}\feqvector{a_1,a_2,a_3,a_4} = \dots \\  
      e^{At}x(0) = \feqvector{a_1cos(2t)-a_2sin(2t), a_1sin(2t) + a_2cos(2t), e^{6t}(a_3cos(5t)+a_4sin(5t)), e^{6t}(-a_3sin(5t)+a_4cos(5t))}
\end{aligned}
$$
Найдём выход системы:
$$
y_{free}(t) = Ce^{At}x(0) = \feqvector[&]{c_1, c_2, c_3, c_4}\feqvector{a_1cos(2t)-a_2sin(2t), a_1sin(2t) + a_2cos(2t), e^{6t}(a_3cos(5t)+a_4sin(5t)), e^{6t}(-a_3sin(5t)+a_4cos(5t))}
$$
$$
    \begin{aligned}
        y_{free}(t) = a_1c_1cos(2t) - c_1a_2sin(2t) + a_1c_2sin(2t) + c_2a_2cos(2t) + \\
        e^{6t}c_3(a_3cos(5t) + a_4sin(5t)) +e^{6t}c_4(a_4cos(5t) - a_3sin(5t))
    \end{aligned}
$$
Сгруппируем слагаемые несколько иным образом:
$$
    \begin{aligned}
        y_{free}(t) =  (a_1c_2 - c_1a_2)sin(2t) ( a_1c_1 + c_2a_2)cos(2t) + \\
        e^{6t}cos(5t)(c_3a_3 + c_4a_4) + e^{6t}sin(5t)(c_3a_4 - a_3c_4)
    \end{aligned}
$$
Посмотрим на желаемый выходной сигнал, и тогда предъявим следующие требования к коэффициентам:
$$
\begin{cases}
    a_1c_2 = a_2c_1 \\
    a_2c_2 + a_1c_1 = 1 \\
    a_3c_3 + a_4c_4 = 1 \\
    a_4c_3 = a_3c_4 
\end{cases}
$$
Как можно заметить, мы не сможем найти эти восемь коэффициентов единственным образом, здесь существуте бесконечное множество их комбинаций, поэтому просто выберем удобную:
$$
    C = \feqvector[&]{\frac{1}{5}, \frac{2}{5}, \frac{3}{25}, \frac{4}{25}}, \tab x(0) = \feqvector{1, 2, 3, 4}
$$
Мы смогли получить параметры $A, C, x(0)$, теперь подставим их в блок $State-Space$ симулинка, а после сравним с заданием $g_w(t)$ через матлабовскую функцию:
\begin{figure}[ht]
    \centering
    \includegraphics[width=1\textwidth]{output_task3.png}
	\caption{Симумляция - сравнение нашего сигнала с желаемым}
\end{figure}
Они совпали, это прекрасно!

\endinput
\chapter{Слежение для системы с астатизмом первого порядка (И-регулятор)}
\label{ch:chap4}

Теперь будем работать с интегральным регулятором следующего вида:
$$
H(s) = \frac{k}{s}
$$


\section{Стационарный режим работы}
Будем пытаться угнаться за сигналом $g(t) = A = 2$.
Выберем следующие $k$:
$$
k_1 = 0.05, k_2 = 0.5,  k_3 = 0.3
$$

Аналитически определим $e_{final}$ пользуясь теоремой о предельном значении для каждого $k$:
$$
\lim_{t\to\infty} y(t) = \lim_{s\to 0}sE(s)
$$
Для начала найдём образ ошибки слежения: $E(s) = W_{g\to e}(s)G(s)$
$$
W_{g\to e} = \frac{1}{1+W(s)} = \frac{s(s^2 +2.5s + 1)}{s(s^2 + 2.5s + 1) + 3k}
$$
$$
G(s) = \frac{A}{s}
$$
$$
E(s) = \frac{A(s^2 +2.5s + 1)}{s(s^2 + 2.5s + 1) + 3k}
$$
В итоге:
$$
\begin{aligned}
  \lim_{t\to\infty} y(t) = \lim_{s\to 0}s\frac{As^2(s^2 +2.5s + 1)}{s(s^2 + 2.5s + 3k + 1)} = 0\\
\end{aligned}
$$
А это значит, что наш астатизм справился с постоянным сигналом, успеваем.

\begin{figure}[ht]
  \centering
  \includegraphics[width=0.8\textwidth]{output_task4_exp1.png}
\caption{Симуляция - стационарный, $k=0.05$}
\end{figure}

\newpage
\begin{figure}[ht]
  \centering
  \includegraphics[width=0.8\textwidth]{output_task4_exp2.png}
\caption{Симуляция - стационарный, $k=0.5$}
\end{figure}

\begin{figure}[ht]
  \centering
  \includegraphics[width=0.8\textwidth]{output_task4_exp3.png}
\caption{Симуляция - стационарный, $k=0.3$}
\end{figure}

\newpage
\begin{figure}[ht]
  \centering
  \includegraphics[width=0.8\textwidth]{output_task4_exp4.png}
\caption{Симуляция - стационарный, сравнение сигналов}
\end{figure}

\begin{figure}[ht]
  \centering
  \includegraphics[width=0.8\textwidth]{output_task4_exp5.png}
\caption{Симуляция - стационарный, сравнение ошибок}
\end{figure}

\newpage
\textbf{Выводы:} при увеличении $k$ начальная амплитуда сигнала $y(t)$ будет больше, а также увеличится время переходного процесса, но рано или поздно мы придём(полностью совпадать) к $g(t)$. 
При небольших $k$ мы приходим к цели быстрее и с мягкими колебаниями.

\section{Движение с постоянной скоростью}
Будем пытаться угнаться за сигналом $g(t) = Vt = 2t$ в случае моего варианта.
Выберем следующие $k$:
$$
k_1 = 0.5, k_2 = 0.05,  k_3 = 0.1
$$

Аналитически определим $e_{final}$ пользуясь теоремой о предельном значении:
$$
W_{g\to e} = \frac{1}{1+W(s)} = \frac{s(s^2 +2.5s + 1)}{s(s^2 + 2.5s + 1) + 3k}
$$
$$
G(s) = \frac{V}{s^2}
$$
Образ ошибки слежения:
$$
E(s) = \frac{Vs(s^2 +2.5s + 1)}{s^2(s(s^2 + 2.5s + 1) + 3k)}
$$
В итоге:
$$
\begin{aligned}
  \lim_{t\to\infty} y(t) = \lim_{s\to 0}s\frac{Vs^2(s^2 +2.5s + 1)}{s^2(s(s^2 + 2.5s + 1) + 3k)} =  \\
  \lim_{t\to\infty} y(t) = \lim_{s\to 0}s\frac{V(s^2 +2.5s + 1)}{s(s^2 + 2.5s + 1) + 3k)} =  \frac{V}{3k} =  \frac{2}{3k}
\end{aligned}
$$
Посчитаем ошибки для каждого $k$:
$$
\begin{aligned}
  e_1 = 4/3 \\
  e_2 = 40/3\\
  e_3 = 20/3
\end{aligned}
$$

\begin{figure}[ht]
  \centering
  \includegraphics[width=0.8\textwidth]{output_task4_exp6.png}
\caption{Симуляция - постоянная скорость, $k=0.5$}
\end{figure}

\newpage
\begin{figure}[ht]
  \centering
  \includegraphics[width=0.8\textwidth]{output_task4_exp7.png}
\caption{Симуляция - постоянная скорость, $k=0.05$}
\end{figure}

\begin{figure}[ht]
  \centering
  \includegraphics[width=0.8\textwidth]{output_task4_exp8.png}
\caption{Симуляция - постоянная скорость, $k=0.1$}
\end{figure}

\newpage
\begin{figure}[ht]
  \centering
  \includegraphics[width=0.8\textwidth]{output_task4_exp9.png}
\caption{Симуляция - постоянная скорость, сравнение сигналов}
\end{figure}

\begin{figure}[ht]
  \centering
  \includegraphics[width=0.8\textwidth]{output_task4_exp10.png}
\caption{Симуляция - постоянная скорость, сравнение ошибок}
\end{figure}

\newpage
\textbf{Выводы:} Так как система имеет у нас астатизм первого порядка, то и линейный сигнал она должна "почти" догонять, с точностью до установившейся ошибки,
что тоже неплохо, в нашем случае $e_{final} = \frac{2}{3k}$. 

Как можно заметить по общему графику ошибок - чем больше $k$, тем $y(t)$ ближе к $g(t)$, и также больше колебаний. И наоборот - чем меньше $k$, тем мы дальше будет от $g(t)$, но с куда меньшими колебаниями.


\section{Движение с постоянным ускорением}

Будем пытаться угнаться за сигналом $g(t) = \frac{at^2}{2} = 0.5t^2 = Bt^2$ в случае моего варианта.

Выберем следующие $k$:
$$
k_1 = 0.05, k_2 = 0.1,  k_3 = 0.5
$$

Определим установившуюся ошибку, пользуясь теоремой о предельном значении:
$$
W_{g\to e} = \frac{1}{1+W(s)} = \frac{s(s^2 +2.5s + 1)}{s(s^2 + 2.5s + 1) + 3k}
$$
$$
G(s) = \frac{2B}{s^3}
$$
Образ ошибки слежения:
$$
E(s) = \frac{2Bs(s^2 +2.5s + 1)}{s^2(s(s^2 + 2.5s + 1) + 3k)}
$$
В итоге:
$$
\begin{aligned}
  \lim_{t\to\infty} y(t) = \lim_{s\to 0}s\frac{2Bs^2(s^2 +2.5s + 1)}{s^3(s(s^2 + 2.5s + 1) + 3k)} =  \\
  = \lim_{s\to 0}s\frac{2B(s^2 +2.5s + 1)}{s(s(s^2 + 2.5s + 1) + 3k))} = \frac{2B}{0} = \infty
\end{aligned}
$$

А значит при любом $k$ ошибка будет улетать в бесконечность.. сигналы $y(t)$ и $g(t)$ будут расходиться

\begin{figure}[ht]
  \centering
  \includegraphics[width=0.8\textwidth]{output_task4_exp11.png}
\caption{Симуляция - постоянное ускорение, $k=0.05$}
\end{figure}

\newpage
\begin{figure}[ht]
  \centering
  \includegraphics[width=0.8\textwidth]{output_task4_exp12.png}
\caption{Симуляция - постоянное ускорение, $k=0.1$}
\end{figure}

\begin{figure}[ht]
  \centering
  \includegraphics[width=0.8\textwidth]{output_task4_exp13.png}
\caption{Симуляция - постоянное ускорение, $k=0.5$}
\end{figure}

\newpage
\begin{figure}[ht]
  \centering
  \includegraphics[width=0.8\textwidth]{output_task4_exp14.png}
\caption{Симуляция - постоянное ускорение, сравнение сигналов}
\end{figure}

\begin{figure}[ht]
  \centering
  \includegraphics[width=0.8\textwidth]{output_task4_exp15.png}
\caption{Симуляция - постоянное ускорение, сравнение ошибок}
\end{figure}

\newpage
\textbf{Выводы:} сигналы будут расходиться, но при этом график ошибки возрастает линейно, и чем больше у нас $k$, тем плавнее и дольше будет происходить отдаление на бесконечность (и наоборот).

\endinput
\chapter{Синтез $\mathcal{H}_\infty$-регулятора по выходу}
\label{ch:chap5}
\section{Условие задачи}

\begin{itemize}
    \item  Рассмотреть математическую модель «тележки» и для одного из наборов матриц $(C_Z,D_Z)$, выполнить следующие шаги:
    \item \item  Задаться не менее, чем двумя значениями ограничивающего параметра $\gamma > 0$.
    Постараться выбрать так, чтобы одно из этих значений было приближенным к минимальному, при котором задача еще будет иметь решение. 
    Для каждого из выбранных $\gamma$:
    \begin{itemize}
        
        \item Синтезировать соответствующий $\mathcal{H}_\infty$-регулятор вида $u = K \hat{x}$ по выходу.
        \item Синтезировать соответствующий $\mathcal{H}_\infty$-наблюдатель.
        \item Найти передаточную функцию (матрицу) Ww→z(s) замкнутой системы от внешнего возмущения $w$ к регулируемому выходу $z$.
       \item Построить для $W_{w\rightarrow z}(s)$ графики покомпонентных АЧХ.
       \item Построить для $W_{w\rightarrow z}(s)$ график сингулярных чисел.
       \item Найти $\mathcal{H}_2$ и $\mathcal{H}_\infty$ нормы  $W_{w\rightarrow z}(s)$ .
       \item Задаться не менее, чем двумя вариантами гармонического внешнего возмущения
        w на основании полученных графиков АЧХ и сингулярных чисел $W_{w\rightarrow z}(s)$. 
        Среди выбранных возмущений должен присутствовать случай, близкий к «наихудшему» и ощутимо отличающийся от него по частоте.
        \item Для каждого из выбранных вариантов внешнего возмущения $w$ выполнить моделирование и построить графики компонент регулируемого выхода $z(t)$.
        \item Сравнить полученные результаты для различных вариантов внешнего возмущения, сделать выводы.
    \end{itemize}
    \item Сравнить полученные результаты для различных вариантов ограничивающего па
    раметра $\gamma$ и сделать выводы.
\end{itemize}

\section{Решение задачи}


Теперь cинтезируем $\mathcal{H}_\infty$-регулятор по выходу, отличаться он как минимум будет тем, что \text{separation principle} в нём не выполняется
и два уравнения Риккати нужно решать вместе:
$$
\begin{cases}
    A^T \mathbf{Q} + \textbf{Q}A + C^T_Z C_Z - \textbf{Q}B(D^T_Z D_Z)^{-1} B^T \textbf{Q}  + \gamma^{-2}QB_w B^T_w \textbf{Q}= 0, \\
    K = -(D^T_Z D_Z)^{-1} B^T \textbf{Q}, \\
    A\textbf{P} + \textbf{P}A^T + B_w B_w^T - \textbf{P} C^T (D_w D_w^T)^{-1} C \textbf{P} + \gamma^{-2}\textbf{P} C^T_Z C_Z \textbf{P} = 0, \\
    L = -\textbf{P}(I - \gamma^2 \textbf{Q} \textbf{P})^{-1} (C + \gamma^{-2}D_w B_w^T \textbf{Q})^T (D_w D_w^T)^{-1}
    
\end{cases} 
$$

Помимо отсутствия \text{separation principle}, теперь нужно также проверять дополнительное условия согласованности, 
мы будем по-умолчанию при каждом синтезе проверять:
$$
    max(\sigma(PQ)) < \gamma^2
$$
% Если $B_w D_w^T = 0$, $D^T_Z D_Z$ - обратима, а также пары $(A, B_w)$ - стабилизируема, 
% $(C_Z, A)$ - обнаруживаема, то существует решение $Q > 0$ уравнения Риккати, и соответствующий наблюдатель
% имеет устойчивую динамику ошибки, также соответствующая ёй $\mathcal{H}_\infty$-норма минимальна.

\newpage
\subsection{Первый набор $(C_{Z1},D_{Z1})$}

Выберем $\gamma = 20$ и получим следующую матрицу регулятора и матрицу коррекции наблюдателя: 
$$
    K = \begin{bmatrix}
        -0.5 & -1.01 \\
    \end{bmatrix}, \tab 
    L = \begin{bmatrix}
        -1.45 \\
        -1.04 \\
    \end{bmatrix}
$$
Получим следующую передаточную матрицу системы:
$$
    W_{w\rightarrow z}(s) = \begin{bmatrix}\frac{1}{1s^{2} + 1s + 0.5} &  \frac{-2s^{3} - 3s^{2} - 2s - 0.5}{1s^{4} + 2s^{3} + 2s^{2} + 1s + 0.25} \end{bmatrix}^T
$$


\begin{figure}[ht]
    \centering
    \includegraphics[width=0.8\textwidth]{freq_ampl_components9.png}
    \caption{Покомпонентные АЧХ}
  \end{figure}

\begin{figure}[ht]
  \centering
  \includegraphics[width=0.8\textwidth]{singular_values9.png}
  \caption{Сингулярные числа}
\end{figure}
Нормы будем считать по следующим лекционным формулам, однако в \text{MATLAB} 
есть готовые реализации через функцию \text{norm}:
$$
    ||W||_{\mathcal{H}_2}  \approx 2.00
$$

$$
    ||W||_{\mathcal{H}_\infty}  \approx 3.09
$$

Выберем хорошую и плохую частоту $f_1, f_2$. 
Хорошей частотой для нас будет являться та, которая меньше увеличивает сигнал по амплитуде, и наоборот.
$$
    f_1 = 5.2 Hz, \tab f_2 = 0.51 Hz
$$

\subsection{Первое гармоническое возмушение}
\begin{figure}[ht]
    \centering
    \includegraphics[width=0.8\textwidth]{z17.png}
    \caption{Моделирование -  регулируемый выход $z(t)$}
  \end{figure}

\newpage
\subsection{Второе гармоническое возмушение}
\begin{figure}[ht]
    \centering
    \includegraphics[width=0.8\textwidth]{z18.png}
    \caption{Моделирование -  регулируемый выход $z(t)$}
  \end{figure}


  Теперь вручную минимизируем $\gamma = 6$ и получим следующую матрицу регулятора и матрицу коррекции наблюдателя: 
  $$
      K = \begin{bmatrix}
        -0.53 & -1.09 \\
    \end{bmatrix}, \tab 
      L = \begin{bmatrix}
        -2.1 \\
        -1.71 \\
    \end{bmatrix}
  $$
  Получим следующую передаточную матрицу системы:
  $$
      W_{w\rightarrow z}(s) = \begin{bmatrix}\frac{1}{1s^{2} + 1.09s + 0.53} &  \frac{-2.18s^{3} - 3.45s^{2} - 2.32s - 0.56}{1s^{4} + 2.18s^{3} + 2.25s^{2} + 1.16s + 0.28} \end{bmatrix}^T
  $$
  
  
  \begin{figure}[ht]
      \centering
      \includegraphics[width=0.8\textwidth]{freq_ampl_components10.png}
      \caption{Покомпонентные АЧХ}
    \end{figure}
  
  \begin{figure}[ht]
    \centering
    \includegraphics[width=0.8\textwidth]{singular_values10.png}
    \caption{Сингулярные числа}
  \end{figure}
  Нормы будем считать по следующим лекционным формулам, однако в \text{MATLAB} 
  есть готовые реализации через функцию \text{norm}:
  $$
      ||W||_{\mathcal{H}_2}  \approx 2
  $$
  
  $$
      ||W||_{\mathcal{H}_\infty}  \approx 2.97
  $$
  
  Выберем хорошую и плохую частоту $f_1, f_2$. 
  Хорошей частотой для нас будет являться та, которая меньше увеличивает сигнал по амплитуде, и наоборот.
  $$
      f_1 = 3.5 Hz, \tab f_2 = 0.44 Hz
  $$
  
  \subsection{Первое гармоническое возмушение}
  \begin{figure}[ht]
      \centering
      \includegraphics[width=0.8\textwidth]{z19.png}
      \caption{Моделирование -  регулируемый выход $z(t)$}
    \end{figure}
  
  \newpage
  \subsection{Второе гармоническое возмушение}
  \begin{figure}[ht]
      \centering
      \includegraphics[width=0.8\textwidth]{z20.png}
      \caption{Моделирование -  регулируемый выход $z(t)$}
    \end{figure}
  

Амплитуда регулируемого выхода $z(t)$ различаются по своему абсолютному значению в зависимости от выбранной частоты
внешних возмущений, наблюдатель в целом показывает те же результаты, что и в прошлом задании мы получили от наблюдения непосредственно за объектом. 
При синтезе такого наблюдателя мы гарантировали $||W||_{\mathcal{H}_\infty} \leq \gamma$,  
также как и в прошлых экспериментах - пиковое сингулярное число системы действительно равняется $||W||_{\mathcal{H}_\infty}$.


\subsection{Вывод}
В этом задании мы синтезировали $\mathcal{H}_\infty$-регулятор по выходу, который нам позволил 
уменьшить значительно "приглушить" АЧХ передаточной матрицы системы в самом нежелательном месте, поэтому в результате мы получили приглушение амплитуды 
лишь в "плохих" частотах у выбранного $z(t)$.
\endinput
\chapter{Общие выводы}
\label{ch:chap6}

В этой работе были рассмотрены некоторые объекты управления, внутри которых скрывались типовые звенья с определёнными параметрами, описывающими их физическую суть. 
Чтобы изучить каждый из объектов мы нашли его передаточную функцию, и с помощью неё смогли посмотреть на временные характеристики системы - её поведение при импульсном и ступенчатом воздействии. 
После мы взглянули на частотную передаточную функцию, и смогли узнать частотные характеристики системы - АЧХ, ФЧХ в линейном и Логарифмическом масштабе. Эти результаты были проделаны с помощью моделирования $\textrm{Matlab}$ и аналитических расчётов.

Использовал связку \textit{Live-script + Matlab}, там же можно взглянуть на графики и код, в \href{https://github.com/GreedlyCore/control_theory_course}{репозитории} можно найти исходники. 
\endinput

% \printbibliography[title=Список использованных источников] % Автособираемый список литературы

\end{document}