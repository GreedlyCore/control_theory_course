\chapter{Исследование управляемости по выходу}
\label{ch:chap5}
\section{Условие задачи}

Необходимо рассмотреть систему:
$$
  \begin{cases}
    \dot{x} = Ax + Bu \\
    y = Cx + Du
  \end{cases}
$$ и выполнить следующие шаги:

\begin{itemize}
    \item Найти Жорданову (или диагональную) форму системы.
    \item Определить управляемость и наблюдаемость каждого из собственных чисел и системы в целом.
    \item Найти матрицу управляемости системы по выходу при $D = \mathbf{0}_{2×2}$, определить её
    ранг, сделать вывод об управляемости системы по выходу.
    \item Проанализировав полученные результаты, попытаться сделать выводы о причинах 
    управляемости или неуправляемости системы по выходу.
    \item Предложить такую матрицу cвязи $D$, которая могла бы обеспечить 
    полную управляемость по выходу.
\end{itemize}

\section{Решение задачи}

Параметры для объекта:
$$
  A = \begin{bmatrix}
  3 & -6 & 4 \\
  4 & -5 & 4 \\
  -4 & 4 & -5 
  \end{bmatrix} \tab
  B = \begin{bmatrix}
    -1 \\ 3 \\ 1 
  \end{bmatrix} \tab
  C = \begin{bmatrix}
    0 & -4 & -3 \\
    0 & -8 & -6
  \end{bmatrix}
$$

\subsection{Исследование управляемости и наблюдаемости системы}


Найдём Жорданову форму системы, в общем виде она выглядит следующим образом:
$$
    \begin{cases}
      \dot{\hat{x}} = P^{-1}\boldsymbol{A}P \hat{x} + P^{-1}\boldsymbol{B} u \\
      y = \boldsymbol{C}P\hat{x} + Du
    \end{cases}
$$
В нашем случае жорданова клетка и входное/выходное воздействие таково:
$$
    \mathbf{A} = \begin{bmatrix}
        -1 & 0 & 0 \\
        0 & -3 & -2 \\
        0 & 2 & -3 
        \end{bmatrix} \tab 
    P^{-1}B = B^* = \begin{bmatrix}
        4 \\ -1.5+1.5i \\ -1.5-1.5i
        \end{bmatrix}
$$
$$
    CP= \begin{bmatrix}
        -3 & 1 & 1 \\
        -6 & 2 & 2 
        \end{bmatrix}
$$
Как можно заметить, три собственных числа соответствуют различным жордановым клеткам, и для каждой из них строка матрицы входных/выходных воздействий не равна нулю.
Значит эти три собственных числа управляемы и наблюдаемы, и как следствие - вся система полностью управляема по состоянию и наблюдаема.





Найдем матрицу управляемости системы по выходу при $D = \mathbf{0}_{2×2}$ :

$$
    U_{out} = \begin{bmatrix}
      CU & D
    \end{bmatrix} =  \begin{bmatrix}
      -15 & 27 & -63 & 0 & 0 \\
      -30 & 54 & -126 & 0 & 0 
    \end{bmatrix}
$$
Нетрудно заметить, что\dots
$$
  rank(U_{out}) = 1
$$
Ранг матрицы управляемости по выходу не равен размерности выхода, в таком случае наша система не управляема по выходу.

Это произошло из-за того, что матрица матрица наблюдения $C$ содержит в себе два линейнозависимых вектора-строки, которые снижают ранг до 1.
Также из-за этого мы теряем информацию по выходу $y(t)$, потому что обе компоненты вектора станет созависимыми и мы не сможем обеспечить все возможные выходы у системы.

Чтобы исправить это и сделать систему управляемой по выходу, необходимо подобрать такую матрицу $D$, которая сделает ранг $U_{out}=2$. 
Например: 
$$
    D = \begin{bmatrix}
        1 & 0 \\
        0 & 1
    \end{bmatrix}
$$
Тогда матрица управляемости по выходу будет равна:
$$
    U^*_{out} = \begin{bmatrix}
      CU & D
    \end{bmatrix} =  \begin{bmatrix}
      -15 & 27 & -63 & 1 & 0 \\
      -30 & 54 & -126 & 0 & 1 
    \end{bmatrix}
$$
и её ранг уже будет равен 2.

\subsection{Вывод}

В этом задании мы рассмтрели полная линейную систему. Мы нашли жорданову форму систему, по которой мы исследовали управляемость по состоянию и наблюдаемость, 
вышло, что система управляема по состоянию и наблюдаема. Однако при этом нулевая матрица связи $D_{2×2}$ не делала эту систему упроавляемой по выходу, но
мы нашли подходящую.
\endinput