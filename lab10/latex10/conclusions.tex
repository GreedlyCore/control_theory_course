\chapter{Общие выводы}
\label{ch:chap6}

В ходе выполнения лабораторной работы мы рассмотрели синтез LQR-регулятора и наблюдателя в виде LQE/Фильтра Калмана, которые двойственны и выбор зависит от характера внешних воздействий / шума (детерминированность/случайность).

Синтезированные компоненты системы проверялись при помощи компьютерного моделирования, наблюдатели успешно сходились к истинной системе только в том случае, когда мы точно знали характер распределения внешних воздействий / шумов, 
в противном случае наблюдатель мог ложно сойтись или не сойтись до нулевой ошибки в целом, колебаясь в некоторых пределах.

В случае регулятора успешно приводил в положение равновесия во всех случаях с разной степенью энергозатрат и оптимальности в виде сочетания перерегулирования между временем переходного процесса.

В случае связки \text{регулятора + наблюдателя = LQG} мы не смогли получить качественную сходимость и устойчивость, потому что не смогли в должной степени понять влияние внешних $f, \xi$, в итоге не добились правильной настройки и регулятора и наблюдателя.

Использовал связку \textit{Live-script + Matlab}, все исходные материалы, использованные в работе можно найти  в \href{https://github.com/GreedlyCore/control_theory_course}{репозитории}.

\endinput