\chapter{Общие выводы}
\label{ch:chap4}
В этой работе мы познакомились со свободным движением, самым простым и очевидным способом управлять(миром) системами - просто отпустить её и ничего не делать!

В третьем задании мы посмотрели на обратную задачу управления - нам дают в руки конкретный двигатель, который управляется такой-то математической функцией, а мы в свою очередь некоторыми методами нашли параметры системы, однозачно описывающие её, теперь по этим матрицам мы сможем делать какие-то выводы о системе.

В первом и втором задании мы посмотрели на основные типы устойчивости при свободном движение, и поняли, что пространство параметров позволяет нам гибко смотреть на систему и управлять её.
\endinput