\chapter{Выводы}
\label{ch:chap5}

В процессе выполнения данной работы мы посмотрели как создавать простейшие линейные системы в связке \textit{MATLAB}+\textit{Simulink}, используя только элементарные блоки(integrator, gain, sum). Посмотрели мы самые разные системы - начиная с 
систем типа "вход-выход", потом преобразовали их в три разновидности "вход-состояния-выход", и убедились, что они дают одни и те же результаты симулирования, 
потому что это разный "взгляд" на линейную систему.

Мы подтвердили следующее предположение: форма В-В может быть представлена бесконечным количеством различных форм В-С-В.

\endinput